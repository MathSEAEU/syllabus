%%%%%%%%%%%%%%%%%%%%%%%%%%%%%%%%%%%%%%%%%
% Sylalbus for the SEA-EU school in Computational and Numerical Mathematics
%
% Inzane Syllabus Template
% LaTeX Template
% Version 1.2 (8.22.2019)
%
% This template has been downloaded from:
% http://www.LaTeXTemplates.com
%
% Original author:
% Carmine Spagnuolo (cspagnuolo@unisa.it) with major modifications by 
% Zane Wolf (zwolf.mlxvi@gmail.com)
%
% I (Zane) have left a lot of instructions both in the .tex file and the .cls file that can guide you to customize this document to suite your tastes and requirements. Here is a brief guide: 
%  - Changing the Main Color: .cls line 39
%  - Adding more FAQs: .cls line 126 and .tex line 99
%  - Adding TA emails: uncomment .cls lines 220 & 224 and .tex lines 85 and 89
%  - Deleting the FAQ sidebar entirely: .tex line 188
%  - Removing the Lab/TA Info and placing a brief Overview/About section in their place:        uncomment .tex line 91 and .cls line 227, and comment .cls lines for the LAB/TA info        that you no longer want (c. lines 184-227)

%
% I am also happy to help with crafting/designing modifications to this template to help suite your personal needs in a syllabus. Feel free to reach out! 
%
% License:
% The MIT License (see included LICENSE file)
%
%%%%%%%%%%%%%%%%%%%%%%%%%%%%%%%%%%%%%%%%%

%----------------------------------------------------------------------------------------
%	PACKAGES AND OTHER DOCUMENT CONFIGURATIONS
%----------------------------------------------------------------------------------------

\documentclass[letterpaper]{inzane_syllabus} % a4paper for A4

\usepackage{booktabs, colortbl, xcolor}
\usepackage{tabularx}
\usepackage{enumitem}
\usepackage{ltablex} 
\usepackage{multirow}

\setlist{nolistsep}

\usepackage{lscape}
\newcolumntype{r}{>{\hsize=0.9\hsize}X}
\newcolumntype{w}{>{\hsize=0.6\hsize}X}
\newcolumntype{m}{>{\hsize=.9\hsize}X}

\renewcommand{\familydefault}{\sfdefault}
\renewcommand{\arraystretch}{2.0}
%----------------------------------------------------------------------------------------
%	 PERSONAL INFORMATION
%----------------------------------------------------------------------------------------

\profilepic{Split_malla.png} % Profile picture, if the height of the picture is less than that of the cirle, it will have a flat bottom. 


% To remove any of the following, you need to comment/delete the lines in the .cls file (c. line 186). Commenting/deleting the lines below will produce an error. 

%To add different lines, you will need to create the new command, e.g. \profPhone, in the .cls file c. line 76, and command to create the line in the side bar in the .cls file c. line 186

\classname{SEA-EU School \\[0.25em] on Computational \\[0.1em] Mathematics} 
\classnum{Split, 9-13 2024} 

%%%%%%%%%%%%%%% SCHOOL  INFO
\prereq{Oriented to: Master or PhD students in Mathematics, Science or Engineering}
\prereqTwo{Prereqisites: \emph{Mathematics} and \emph{computer programming} skills}
\classdays{Sept 2004: 2-6 sept (online) $+$ 9-13 (face to face)}
\classloc{University of Split, Croatia}

\course{Introduction to partial differential equations: examples and numerical resolution by the finite element method}
\courseTwo{Numerical Optimization}
\courseThree{Neural Networks for PDE}

\conference{Brief title of first conference}
\conferenceTwo{Brief title of second conference}
\conferenceThree{Brief title of third conference}
\conferenceFour{Brief title of fourth conference}


%%%%%%%%%%%%%%% LAB INFO
\labloc{\url{http://????}}

% %%%%%%%%%%%%%%% TA INFO
% \taAname{Alice}
% \taAofficehours{Office Hrs: Tues \& Thurs 10-11a}
% \taAoffice{MCZ 104}
% % \taAemail{}
% \taBname{James}
% \taBofficehours{Office Hrs: Tues \& Thurs 3-4p}
% \taBoffice{MCZ 104}
% \taBemail{}

% \about{Fish make up the largest group of vertebrates on the planet, easily outnumbering mammals, marsupials, birds, and reptiles combined. Not only are they abundant, but they've diversified into an extraordinary array of sizes, shapes, lifestyles, and habitats. You can find them in the coldest, deepest parts of the ocean, and in the hottest freshwater ponds in the desert. This course will explore fish diversity and their biology. } 


%---------------------------------------------------------------------------------------
%	 FAQs
%----------------------------------------------------------------------------------------
%to add more questions or remove this section, go to the .cls file and start with lines comment
%lines 226-250. Also comment out this section as well as line 152(ish), the command \makeSide

\qOne{Do we need deep skills in Mathematics?}
\aOne{Yes, we do actually dissect fish. If you know of any issues that may cause you difficulties during dissections, please notify your TA ASAP.}

\qTwo{Do we need deep skills in Computer Programming?}
\aTwo{No clue. When someone says `fish', we have a picture of a general fish of a general shape in our minds, but the truth is that `fish' doesn't have scientific meaning. Here's a funny video about that: \href{https://youtu.be/uhwcEvMJz1Y}{Youtube (hyperlink)}. }

\qThree{What is your favorite fish?}
\aThree{A lumpsucker. They are incredibly, adorably weird-looking.}

\qFour{What's the difference between plural `fish' and `fishes'?}
\aFour{`Fish' is the plural form when talking about two or more fish of the same species. `Fishes' is the plural when talking about two or more different species.}

%----------------------------------------------------------------------------------------

\begin{document}

%----------------------------------------------------------------------------------------
%	 DESCRIPTION
%----------------------------------------------------------------------------------------

\makeprofile % Print the sidebar

%----------------------------------------------------------------------------------------
%	 OVERVIEW
%----------------------------------------------------------------------------------------
\section{Overview}
 
This school is devoted to introducing young students interested in mathematical and computational modeling into some of the most trending techniques being used today in this field. With special interest in practical application to science and engineering of models formulated in terms of partial differential equations (PDEs).

It is structured around three courses. Two of them focus on computing numerical solutions of PDEs. The widely used finite element method (FEM) will be studied along to other emerging techniques such as physically informed neural networks (PINN).
Numerical methods for constrained and unconstrained optimization will also be covered in a third course. 
%The school will be structured around three courses. where, in order to approach the solution of PDEs we will combine some of the most successful techniques used in the last years and other emerging techniques such as physically informed neural networks.
Throgh several examples and using high level open source computer tools, we explore algorithms for optimization, formulate and solve numerically PDE models and post-process the solutions for high-quality plotting. 

The required mathematical and computer skills will be standarized in introductory online sessions and the entire learning process will be supervised by professors who are experts in this field. 
The courses are complemented by some conferences where lecturers related to SEA-EU universities show introduce recent research projects related to the subjects of the School.

%It will be structured around three courses where some of the some of the more succesful techniques    the  variational formulation of PDEs and the Finite Element Method (FEM). Using high level open source computer tools, we explore techniques for meshing of spatial domains, numerical solving of systems of equations and post-processing the solutions with high-quality visualization engines. The required mathematical and computer skills will be standardized in the first introductory sessions.

%Topics like thermal diffusion, linear or nonlinear elasticity, migration of biological cells, fluid dynamics in oceanography or naval engineering and many other matters of practical interest will be covered.
%The course is complemented by some conferences in which lecturers related to SEA-EU universities show their experience in recent research projects related to these subjects.


\vspace{0.5cm}
\section{Learning Objectives}

%use \begin{outline} or \begin{outline}[enumerate] to create a list with subitems. 
\begin{itemize}
  \item Formulate some PDEs arising in models from physics and engineering \emph{\color{myCOLOR} [Paco, can you improve these three objectives?]}
  % \item Accurately use standard mathematical notation for the variational formulation of linear elliptic PDEs with Dirichlet or Neumann boundary conditions.
  \item Program 
    scripts for meshing 2 and 3-dimensional domains, solving variational formulations of elliptic equations and plotting the solutions
  % \item Use the \texttt{Paraview} platform for post-processing and plotting variables on meshes. \textbf{Paco??}
  \item Apply the Euler numerical method for mathematical time semi-discretization and computer resolution of the heat equation or other parabolic PDEs
  \item Improve your understanding of classical methods for unrestricted optimization \emph{\color{myCOLOR} [Malte, can you improve these three objectives?]}
  \item Gain skills regarding methods for restricted optimization 
  \item Program computer scripts for testing and comparing restricted and unrestricted optimization algorithms 
  \item Become familiar with the concept of  neural networks (NN) as mathematical and computational objects minimizing a deep cost functional
  \item Understand the particular case of physics informed neural network (PINN) for solving PDE models
  % \item Have a basic working knowledge of NN and PINN and programming computer scripts applying them to convection, diffusion or fluid dinamics 
  \item Program computer scripts for basic NN and apply PINN to solve convection, diffusion or fluid dinamics models

\end{itemize}


\vspace{0.5cm} %I make liberal use of the \vspace{} command to partition and place sections just how I want them. Alter as you see fit. 
\section{Metodology}

\lipsum[1]

%%%%%%%%%%%%%%%%%%%%%%%%%%%%%%%%%%%%%%%%%%%%%%%%%%%%%%%%%%%%%%%%%%%%%%%%%%%%%
%                SECOND PAGE
%%%%%%%%%%%%%%%%%%%%%%%%%%%%%%%%%%%%%%%%%%%%%%%%%%%%%%%%%%%%%%%%%%%%%%%%%%%%%

\newpage % Start a new page

\makeSide % Print the FAQ sidebar; To get rid of, simply comment out and uncomment \makeFullPage

% \makeFullPage

%----------------------------------------------------------------------------------------
%	 READING MATERIAL
%----------------------------------------------------------------------------------------
\vspace{0.5cm} %I make liberal use of the \vspace{} command to partition and place sections just how I want them. Alter as you see fit. 
\section{Material}

{\color{myCOLOR} Recommended Text}\\
Paxton, J.R. \& Eschmeyer, W.N. \textit{Encyclopedia of Fishes}. 2nd Edition. Harcourt Brace \& Co. 1998. \\

{\color{myCOLOR} Software}\\
Helfman, G.S., Collette, B.B., Facey, D.E., \& Bowen, B.W. \textit{The Diversity of Fishes: Biology, Evolution, and Ecology}. 2nd Edition. Wiley-Blackwell. 2009. ("DOF") \\


{\color{myCOLOR} Other}\\
Any required journal articles and book chapters will be provided on Canvas. 

\vspace{0.5cm}
\section{Travel and Accommodation for Students}

\lipsum[2]

\vspace{0.5cm}
\section{Evaluation}

\lipsum[3]

\vspace{0.5cm}
\section{Registration}

\lipsum[4]

\vspace{0.5cm}
\section{Organizing Committee}

Saša Krešić-Jurić (University of Split), Francisco Ortegón Gallego, Victoria Redondo Neble, J. Rafael Rodríguez Galván (University of Cádiz) Malte Braak (University of Kiel), Hermenegildo Borges de Oliveira (University of Algarve). 

%%%%%%%%%%%%%%%%%%%%%%%%%%%%%%%%%%%%%%%%%%%%%%%%%%%%%%%%%%%%%%%%%%%%%%%%%%%%%
%                COURSE SCHEDULE
%%%%%%%%%%%%%%%%%%%%%%%%%%%%%%%%%%%%%%%%%%%%%%%%%%%%%%%%%%%%%%%%%%%%%%%%%%%%%
\newpage
\makeFullPage
\section{Class Schedule}

\begin{center}
\begin{tabularx}{\textwidth}{p{2cm}p{8cm}p{9.5cm}} %change the width of the comments by changing these cm measurements. Add/substract columns by adding/deleting p{} sections. 
\arrayrulecolor{myCOLOR}\hline
%%%%%%%%%%%%%%%%%%%%%%%%%%%%%%%%%%%%%%%%%%% MODULE 1
\multicolumn{3}{l}{\textbf{\textcolor{myCOLOR}{\large MODULE 1: Life's Building Blocks }}} \\
\hline
% Week & Topic & Readings \\ \hline 
%%Alternatively, instead of Week #, you can do Class date for meeting
Week 1 & History of the Earth - Fish Remix & Friedman, M. \& Salland, L.C. (2012). Five hundred million years of extinction and recovery: A Phanerozoic survey of large-scale diversity patterns in fishes. \textit{Palaeontology}, 55(4):707-742 \\

& Stem \& Extant Agnathans \& Gnathostomes & DOF Ch. 11, pp. 169-179; Ch. 13, pp. 231-240  \\
& & Brazeau, M.D. \& Friedman, M. (2015). The origin and early phylogenetic history of jawed vertebrates. \textit{Nature}, 520(7548): 490-497.\\
\arrayrulecolor{maingray}\hline
Week 2 & Chondrichthyans I: Overview \& Sharks & DOF Ch. 11, pp. 197-200; Ch. 12, pp. 205-227\\

& Chondrichthyans II: Batoids \& Chimaeras & DOF Chapter 12, pp. 227-229 \\
\arrayrulecolor{maingray}\hline
Week 3 & Stem \& Extant Sarcopterygians & DOF Ch. 11, pp. 179-185; Ch. 13, pp. 242-248 \\

& Actinopts I: Overview & DOF Ch. 14 \& Ch. 15 \\
\arrayrulecolor{maingray}\hline
Week 4 & Actinopts II: Basal Actinopts \& Teleostei & DOF Ch. 11, pp. 185-197; Ch. 13, pp. 248-259, Ch. 14, pp. 261-266 \\

& Actinopts III: Otocephalan Fishes & DOF Ch. 14, pp. 267-275 \\

\arrayrulecolor{maingray}\hline
Week 5 & Actinopts IV: Freshwater Fishes & DOF Ch. 16, pp. 339-354; Ch. 18, pp. 410-414, 417-421 \\

& Actinopts V: Deep Sea Fishes & DOF Ch. 18, pp. 393-401 \\
& & Davis, M.P., Sparks, J.S., \& Smith, W. L. (2016). Repeated and widespread evolution of bioluminescence in marine fishes. \textit{PLOS One}.\\
\arrayrulecolor{maingray}\hline
Week 6 & Actinopts VI: Coral Reef Fishes & Bellwood, D.R. \& Wainwright, P.C. (2002). The History and Biogeography of Fishes on Coral Reefs. \textit{Coral Reef Fishes: Dynamics and Diversity in a Complex Ecosystem}, 5-32. \\

 & Actinopts VII: Pelagic Fishes & DOF Ch. 18, pp. 401-405 \\
 \arrayrulecolor{maingray}\hline
 Week 7 & Review & Module 1 \\
 &EXAM &  MIDTERM 1 \\
 
 \arrayrulecolor{myCOLOR}\hline
\multicolumn{2}{l}{\textbf{\textcolor{myCOLOR}{\large MODULE 2: What Makes a Fish }}} \\
\hline
 Week 8 & Respiration & DOF Ch. 5 \\
 
 & Cardiovascular Systems & DOF Ch. 4, pp. 45-48 \\
 \arrayrulecolor{maingray}\hline
Week 9 & Homeostasis & DOF Ch. 4, pp. 52; Ch. 7, pp. 101-105.\\

& Feeding Mechanisms & DOF Ch. 4, pp. 41-42; Ch. 8, pp. 119-126  \\
\arrayrulecolor{maingray}\hline
Week 10 & Sensory Systems & DOF Ch. 6 \\

&  Buoyancy & DOF Ch. 4, pp. 50-52  \& Ch. 5, pp. 68-70 \\
\arrayrulecolor{maingray}\hline
Week 11 &  Locomotion I - Undulatory Propulsion &  Webb, P.W. (1984). Form and function in fish swimming. \textit{Sci. Amer.}, 251(1): 72-83. \\
% & & Shadwick, R.E. (2005). How tunas and lamnid sharks swim: An evolutionary convergence. \textit{Amer. Sci.}, 93: 524-531. \\

& Locomotion II - Oscillatory Propulsion & Daniel, T.L. (1984). Unsteady Aspects of Aquatic Locomotion. \textit{Amer. Zoo.}, 24: 121-134.\\
\arrayrulecolor{maingray}\hline
Week 12 & Communication \& Reproduction  &  DOF Ch. 22, pp. 477-485 \\

& & DOF Ch. 21  \\

&Review & Module 2\\
\arrayrulecolor{maingray}\hline
Week 13 & EXAM & MIDTERM 2\\
&Holiday & Thanksgiving \\

\arrayrulecolor{myCOLOR}\hline

\multicolumn{2}{l}{\textbf{\textcolor{myCOLOR}{\large MODULE 3: There Goes the Neighborhood }}} \\
\hline
Week 14 & Symbiotic Relationships & DOF Ch. 22, 492-497 \\

& Behavior & DOF Ch. 23 \\
\arrayrulecolor{maingray}\hline
Week 15 & Ecology & DOF Ch. 25 \\

& Conservation Efforts & DOF Ch. 26 \\
\arrayrulecolor{myCOLOR}\hline
Week 16 & FINAL EXAM & Date \& Time \& Location \\ 
\hline 
\end{tabularx}
\end{center}

%%%%%%%%%%%%%%%%%%%%%%%%%%%%%%%%%%%%%%%%%%%%%%%%%%%%%%%%%%%%%%%%%%%%%%%%%%%%%
%                LAB SCHEDULE
%%%%%%%%%%%%%%%%%%%%%%%%%%%%%%%%%%%%%%%%%%%%%%%%%%%%%%%%%%%%%%%%%%%%%%%%%%%%%
\newpage
\section{Lab Schedule}

\begin{center}
\begin{tabularx}{\textwidth}{p{2cm}p{6.5cm}p{11cm}}
\arrayrulecolor{myCOLOR}\hline
Week 2 & Chondrichthyan Fishes & Students enjoy a two part lab: first, they examine specimens across the Chondrichthyan phylogeny; second, they dissect a small spiny dogfish shark. \\
\arrayrulecolor{maingray}\hline 
Week 3 & Harvard Natural History Museum & Students walk through the HMNH and the fossil collection, inspecting various fossil fishes. \\
\hline 
Week 4 & Basal Teleosts \& Otocephalan Fishes & Students explore specimens across the basal Teleost phylogeny. \\
\hline 
Week 5 & Freshwater \& Deep-Sea Fishes & Students explore specimens from a diverse group of fishes, and try to place each group in the broader phylogeny. \\
\hline 
Week 6 & Coral Reef \& Pelagic Fishes & Students explore specimens from a diverse group of fishes, and try to place each group in the broader phylogeny.\\
\hline 
Week 7 & No Lab & \\ 
\hline 
Week 8 & Internal Systems & Students dissect fish specimens, probing and examing key internal systems. \\
\hline 
Week 9 & Jaw Dissections & Students again dissect their fish specimens, taking apart and visualizing the jaws of their fish. \\
\hline 
Week 10 & Sensory Systems \& Buoyancy & Students again enjoy a two-part lab: first, examining a broad selection of specimens, comparing and contrasting sensory system apparatuses; and then conducting a series of small experiments to better understand the difficulties associated with buoyancy control in the water. \\
\hline 
Week 11 & Locomotion & Students dissect fish specimens, looking at muscular and structure of the body and fins. Students also participate in demonstrations designed to elucidate the concept of lift. \\
\hline 
Week 12 & Review Paper Projects & Students bring electronic devices and/or paper printouts of 2-3 paper choices, and will select peer reviewers. TAs will be available to assist students in choosing a paper and begin reviewing it. \\
\hline 
Week 13 & No Lab & \\
\hline 
Week 14 & No Lab & \\
\hline 
Week 15 & Final Exam Review Sessions & Review Paper Project Due \\
\arrayrulecolor{myCOLOR}\hline

\end{tabularx}
\end{center}

%----------------------------------------------------------------------------------------

\end{document} 




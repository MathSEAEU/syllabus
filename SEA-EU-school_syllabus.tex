%%%%%%%%%%%%%%%%%%%%%%%%%%%%%%%%%%%%%%%%%
% Sylalbus for the SEA-EU school in Computational and Numerical Mathematics
%
% Inzane Syllabus Template
% LaTeX Template
% Version 1.2 (8.22.2019)
%
% This template has been downloaded from:
% http://www.LaTeXTemplates.com
%
% Original author:
% Carmine Spagnuolo (cspagnuolo@unisa.it) with major modifications by 
% Zane Wolf (zwolf.mlxvi@gmail.com)
%
% I (Zane) have left a lot of instructions both in the .tex file and the .cls file that can guide you to customize this document to suite your tastes and requirements. Here is a brief guide: 
%  - Changing the Main Color: .cls line 39
%  - Adding more FAQs: .cls line 126 and .tex line 99
%  - Adding TA emails: uncomment .cls lines 220 & 224 and .tex lines 85 and 89
%  - Deleting the FAQ sidebar entirely: .tex line 188
%  - Removing the Lab/TA Info and placing a brief Overview/About section in their place:        uncomment .tex line 91 and .cls line 227, and comment .cls lines for the LAB/TA info        that you no longer want (c. lines 184-227)

%
% I am also happy to help with crafting/designing modifications to this template to help suite your personal needs in a syllabus. Feel free to reach out! 
%
% License:
% The MIT License (see included LICENSE file)
%
%%%%%%%%%%%%%%%%%%%%%%%%%%%%%%%%%%%%%%%%%

%----------------------------------------------------------------------------------------
%	PACKAGES AND OTHER DOCUMENT CONFIGURATIONS
%----------------------------------------------------------------------------------------

\documentclass[letterpaper]{inzane_syllabus} % a4paper for A4

\usepackage{booktabs, colortbl, xcolor}
\usepackage{tabularx}
\usepackage{enumitem}
\usepackage{ltablex} 
\usepackage{multirow}

\setlist{nolistsep}

\usepackage{lscape}
\newcolumntype{r}{>{\hsize=0.9\hsize}X}
\newcolumntype{w}{>{\hsize=0.6\hsize}X}
\newcolumntype{m}{>{\hsize=.9\hsize}X}

\renewcommand{\familydefault}{\sfdefault}
\renewcommand{\arraystretch}{2.0}
%----------------------------------------------------------------------------------------
%	 PERSONAL INFORMATION
%----------------------------------------------------------------------------------------

\profilepic{img/Split_malla.png} % Profile picture, if the height of the picture is less than that of the circle, it will have a flat bottom. 


% To remove any of the following, you need to comment/delete the lines in the .cls file (c. line 186). Commenting/deleting the lines below will produce an error. 

%To add different lines, you will need to create the new command, e.g. \profPhone, in the .cls file c. line 76, and command to create the line in the side bar in the .cls file c. line 186

\classname{SEA-EU School \\[0.25em] on Computational \\[0.1em] Mathematics} 
\classnum{Split, Sept. 9-13, 2024} 

%%%%%%%%%%%%%%% SCHOOL  INFO
\prereq{Oriented to: Master or PhD students in Mathematics, Science or Engineering}
\prereqTwo{Prerequisites: \emph{Mathematics} and \emph{computer programming} skills}
\classdays{Sept 2004: 2-6 Sept (online) $+$ 9-13 (face to face)}
\classloc{University of Split, Croatia}

\course{Introduction to partial differential equations: examples and numerical resolution by the finite element method}
\courseTwo{Physics-informed neural networks: study on fluid dynamics}
\courseThree{Numerical Optimization}

\conference{Brief title of first lecture}
\conferenceTwo{Brief title of second lecture}
\conferenceThree{Brief title of third lecture}
\conferenceFour{Brief title of fourth lecture}


%%%%%%%%%%%%%%% LAB INFO
\labloc{\url{http://????}}

% %%%%%%%%%%%%%%% TA INFO
% \taAname{Alice}
% \taAofficehours{Office Hrs: Tues \& Thurs 10-11a}
% \taAoffice{MCZ 104}
% % \taAemail{}
% \taBname{James}
% \taBofficehours{Office Hrs: Tues \& Thurs 3-4p}
% \taBoffice{MCZ 104}
% \taBemail{}

% \about{Fish make up the largest group of vertebrates on the planet, easily outnumbering mammals, marsupials, birds, and reptiles combined. Not only are they abundant, but they've diversified into an extraordinary array of sizes, shapes, lifestyles, and habitats. You can find them in the coldest, deepest parts of the ocean, and in the hottest freshwater ponds in the desert. This course will explore fish diversity and their biology. } 


%---------------------------------------------------------------------------------------
%	 FAQs
%----------------------------------------------------------------------------------------
%to add more questions or remove this section, go to the .cls file and start with lines comment
%lines 226-250. Also comment out this section as well as line 152(ish), the command \makeSide

\qOne{Pending task: decide if we should dedicate this page of the sidebar to frequent questions}
\aOne{Pending task, this section has not yet been filled up.}

\qTwo{What skills do i need in mathematics?}
\aTwo{Pending task, this section has not yet been filled up.}

\qThree{What skills do i need in computing?}
\aThree{Pending task, this section has not yet been filled up.}

\qFour{How to apply for Erasmus+ funding?}
\aFour{Pending task, this section has not yet been filled up.}

%----------------------------------------------------------------------------------------

\begin{document}

%----------------------------------------------------------------------------------------
%	 DESCRIPTION
%----------------------------------------------------------------------------------------

\makeprofile % Print the sidebar

%----------------------------------------------------------------------------------------
%	 OVERVIEW
%----------------------------------------------------------------------------------------
\section{Overview}

This school is aimed at young students looking to acquire skills on the most trending techniques in mathematical and computational modeling. We place a strong emphasis in practical applications in science and engineering, with a focus on modeling spatiotemporal variations of physical quantities using partial differential equations (PDEs). And related topics such as numerical optimization, playing a vital role today in fields like deep learning.

The school is structured into three courses. 
Two of them cover the widely used finite element method (FEM) and other emerging techniques such as physically informed neural networks (PINN) for computing numerical solutions of PDEs.
A third course focus on numerical methods for constrained and unconstrained optimization. 
In these courses, we introduce fundamental math structures, provide illustrative examples, review algorithms, program them with high-level open-source tools, and post-process the solutions for high-quality plotting.

The necessary mathematical and computer skills will be standardized in introductory online sessions, with the entire learning process overseen by expert professors in the field.
The courses are complemented by some conferences where lecturers related to SEA-EU universities  introduce recent research projects related to the subjects of the School.



\vspace{0.5cm}
\section{Learning Objectives}

%use \begin{outline} or \begin{outline}[enumerate] to create a list with subitems. 
\begin{itemize}
  \item Formulate some PDEs arising in models from physics and engineering \emph{\color{myCOLOR} [Paco, can you improve these three objectives?]}
  \item Program 
    scripts for meshing 2 and 3-dimensional domains, solving variational formulations of elliptic equations and plotting the solutions
  \item Apply the Euler numerical method for mathematical time semi-discretization and computer resolution of the heat equation or other parabolic PDEs
  \item Improve your understanding of classical methods for unrestricted optimization \emph{\color{myCOLOR} [Malte, can you improve these three objectives?]}
  \item Gain skills regarding methods for restricted optimization 
  \item Program computer scripts for testing and comparing restricted and unrestricted optimization algorithms 
  \item Become familiar with the concept of  neural networks (NN) as mathematical and computational objects minimizing a deep cost functional
  \item Understand the particular case of physics informed neural network (PINN) for solving PDE models
  \item Program computer scripts for basic NN and apply PINN to solve convection, diffusion or fluid dynamics models

\end{itemize}


\vspace{0.5cm} 
\section{Methodology}

The school begins with a phase dedicated to homogenizing the students' knowledge to help them reach a basic level before the classroom courses begin to develop in Split. It will be designed as 5 online days where students will be able to access material specifically prepared for them to acquire basic math and computer skills. During these days they will have the support and tutoring of the curses instructors, through video meetings and forums in a virtual campus.

In a second week, the School will be held physically on the premises of the University of Split. It will consist on three courses where  theoretical lectures will be given by the professor and numerous examples, algorithms and exercises will be reviewed. In computer classes, students will be provided with the necessary software to implement them and analyze the results. The courses will be complemented with conferences where invited professors will stimulate the curiosity of the students, showing recent research topics related with the matter.

After the end of the classroom days in Split, the students interested in evaluation to be recognized up to three credits, will deliver a final work in small groups related to the school topics. For this, they will be assisted by a final tutoring session.

%%%%%%%%%%%%%%%%%%%%%%%%%%%%%%%%%%%%%%%%%%%%%%%%%%%%%%%%%%%%%%%%%%%%%%%%%%%%%
%                SECOND PAGE
%%%%%%%%%%%%%%%%%%%%%%%%%%%%%%%%%%%%%%%%%%%%%%%%%%%%%%%%%%%%%%%%%%%%%%%%%%%%%

\newpage % Start a new page

\makeSide % Print the FAQ sidebar; To get rid of, simply comment out and uncomment \makeFullPage

% \makeFullPage

%----------------------------------------------------------------------------------------
%	 READING MATERIAL
%----------------------------------------------------------------------------------------

\section{Evaluation}

Some students might be interested to be recognized for up to three credits. For that we have foreseen an evaluation process, which will be based on daily attendance to the school and  a small group work. It will be related to the school topics and will be delivered after the end of the classroom sessions.


\vspace{0.5cm}
\section{Travel and Accommodation}

{\color{myCOLOR}Sa\v sa, can you review ?}

The city of Split, Croatia, is served by an international airport which is located approximately 20 km from the city center of Split, on the west side of Kaštela Bay. Regular taxis at Split Airport are available during the airport operation times, being the average cab fare to the center of Split about 30\euro. Other options for travelling include Shuttle Bus (Pleso Prijevoz) and Local Bus, with respective costs of 6\euro. and 2.5\euro. 

The School will provide students with up to 20 places in the university residence located on the university campus, a 5-minute walk from the faculty of science, where the activities will take place. Breakfast and lunch are included.
  
\vspace{0.5cm}
\section{Registration}

\foreach \n in {1,...,12}{Pending task, this section has not yet been filled up. }

\vspace{0.5cm}
\section{Erasmus Funding}

{\color{myCOLOR}Sa\v sa, can you review ?}

Up to 20 Students coming from a EU Member State or third country associated to the Erasmus+ program can apply for funding through  Blended Intensive Program (BIP). This will allow them to get support to attend the classroom week in Split. 

\vspace{0.5cm} %I make liberal use of the \vspace{} command to partition and place sections just how I want them. Alter as you see fit. 
\section{Material and Bibliography}

{\color{myCOLOR}Task pending}: decide if this section should be included in a syllabus.

{\color{myCOLOR} Recommended Text}\\
Paxton, J.R. \& Eschmeyer, W.N. \textit{Encyclopedia of Fishes}. 2nd Edition. Harcourt Brace \& Co. 1998. \\

{\color{myCOLOR} Software}\\
Helfman, G.S., Collette, B.B., Facey, D.E., \& Bowen, B.W. \textit{The Diversity of Fishes: Biology, Evolution, and Ecology}. 2nd Edition. Wiley-Blackwell. 2009. ("DOF") \\

{\color{myCOLOR} Other}\\
Any required journal articles and book chapters will be provided on Canvas. 


\vspace{0.5cm}
\section{Organizing Committee}

Saša Krešić-Jurić (University of Split), Francisco Ortegón Gallego, Victoria Redondo Neble, J. Rafael Rodríguez Galván (University of Cádiz) Malte Braack (University of Kiel), Hermenegildo Borges de Oliveira (University of Algarve). 

%%%%%%%%%%%%%%%%%%%%%%%%%%%%%%%%%%%%%%%%%%%%%%%%%%%%%%%%%%%%%%%%%%%%%%%%%%%%%
%                COURSES
%%%%%%%%%%%%%%%%%%%%%%%%%%%%%%%%%%%%%%%%%%%%%%%%%%%%%%%%%%%%%%%%%%%%%%%%%%%%%
\newpage
\makeFullPage

\newcommand{\block}[3]{\par\textit{#1:} \textbf{#2} #3}

\section{Courses}

\subsection{Course 1. Introduction to partial differential equations: examples and numerical resolution by the finite element method}
\block{Organizer \& Speaker}{Francisco Ortegón Gallego}{(Universidad de Cádiz)}
\block{Classroom time}{}{10 hours}
\block{Topics}{}{}
\begin{enumerate}
  \item Basic notions: differential operators and integral identities. 
  \item Some PDEs arising in physics and engineering.
  \item Variational formulation. Functional spaces. The finite element method.
  \item Introduction fo \texttt{Freefem++}: numerical resolution of PDEs.
  \item Working on a 3D problem: 3D tetrahedralization, resolution and 
  post-processing.
\end{enumerate} 


\subsection{Course 2. Physics-informed neural networks: introduction and case study on fluid dynamics}
\block{Organizers \& Speakers}{J. Rafael Rodríguez Galván, M. Victoria Redondo Neble}{(Universidad de Cádiz)}
\block{Classroom time}{}{10 hours}
\block{Topics}{}{}
% \block{Topics}{}
\begin{enumerate}
  \item Introduction and mathematical foundations of neural networks (NN). Fundamentals, architectures, activation and loss functions, differentiation and chain rule for functions of several variables
  \item Training and optimization in NN. Backpropagation, optimization algorithms
  \item Computational aspects and software libraries. Significance of hardware acceleration and parallelization for efficient training 
  \item A perspective on neural networks for PDE models: physics informed neural networks (PINN). PINN software libraries. Modeling diffusion and convection for linear and non-linear processes
  \item Governing equations in fluid dynamics. The finite element method (FEM) for approximating non-turbulent flows   
  \item PINN in fluid dynamics: comparing to FEM, exploring the pros and cons in a challenging case with applications to real-world scenarios 
\end{enumerate}


\subsection{Course 3. Numerical optimization}
\block{Organizer \& Speaker}{Malte Braack}{(University of Kiel)}
\block{Classroom time}{}{6 hours lectures, 4 hours exercises}
\block{Topics}{}{}
% \block{Topics}{}
\begin{enumerate}
  \item Numerical methods for unrestricted optimization:
  Newton, steepest decent, Armijo step length control
  \item Restricted Optimization Problems:
  equality and inequality constraints, types of restricted optimization problems
  \item Sequential Unrestricted Minimization Technique (SUMT):
  penalty method, SUMT for equality constraints, SUMT for inequality constraints
  \item Stationary points for Restricted Optimization Problems:
  first-order necessary condition, active sets, linearized tangential cones, Abadie constraint qualification,
  Lagrange function, Karush-Kuhn-Tucker (KKT) system, Farkas lemma
  \item Convex optimization and Slater condition:
  convex constraints, relation between Slater condition and KKT
  \item Numerical methods based on Karush-Kuhn-Tucker system:
  Lagrange-Newton for restricted optimization, Sequential Quadratic Programing (SQP)
\end{enumerate} 
%%%%%%%%%%%%%%%%%%%%%%%%%%%%%%%%%%%%%%%%%%%%%%%%%%%%%%%%%%%%%%%%%%%%%%%%%%%%%
%                COURSE CONFERENCES
%%%%%%%%%%%%%%%%%%%%%%%%%%%%%%%%%%%%%%%%%%%%%%%%%%%%%%%%%%%%%%%%%%%%%%%%%%%%%
\newpage
\makeFullPage

\section{Lectures}

\subsection{1. Pending task: Title title title...}
\block{Speaker}{Pending task: Speaker}{(University)}
\block{Classroom time}{1 hour}{}
\block{Abstract}{}{}

\foreach \n in {1,...,12}{Pending task, this section has not yet been filled up. }

%---

\subsection{2. Pending task: Title title title...}
\block{Speaker}{Pending task: Speaker }{(University)}
\block{Classroom time}{1 hour}{}
\block{Abstract}{}{}

\foreach \n in {1,...,12}{Pending task, this section has not yet been filled up. }

%---

\subsection{3. Pending task: Title title title...}
\block{Speaker}{Pending task: Speaker }{(University)}
\block{Classroom time}{1 hour}{}
\block{Abstract}{}{}

\foreach \n in {1,...,12}{Pending task, this section has not yet been filled up. }

%---

\subsection{4. Pending task: Title title title...}
\block{Speaker}{Pending task: Speaker }{(University)}
\block{Classroom time}{1 hour}{}
\block{Abstract}{}{}

\foreach \n in {1,...,12}{Pending task, this section has not yet been filled up. }


%%%%%%%%%%%%%%%%%%%%%%%%%%%%%%%%%%%%%%%%%%%%%%%%%%%%%%%%%%%%%%%%%%%%%%%%%%%%%
%                COURSE SCHEDULE
%%%%%%%%%%%%%%%%%%%%%%%%%%%%%%%%%%%%%%%%%%%%%%%%%%%%%%%%%%%%%%%%%%%%%%%%%%%%%
\newpage
\makeFullPage

\section{Schedule for Online Introduction}
\medskip

{\large September 2-6 2024}
\begin{itemize}
  \item \textbf{Monday 2}, 9:00h. Opening (video conference).
 Introduction to Course 1 (video conference).
 \\
 10:00h \emph{Francisco Ortegón Gallego}. Individual work, tutored in forums on the virtual campus  
\item \textbf{Tuesday 3}, 9:00h. Introduction to Course 2 (video conference). \emph{J. Rafael Rodríguez Galván \& M. Victoria Redondo Neble}. Individual work, tutored in forums on the virtual campus  
    \item \textbf{Wednesday 4}, 9:00h. Introduction to Course 3 (video conference). \emph{Malte Braack}. Individual work, tutored in forums on the virtual campus  
    \item \textbf{Thursday 5} and \textbf{Wednesday 6}. Guided exercises. 
\end{itemize}



\vspace{0.5cm}
\section{Schedule for Classroom Sessions}
\medskip

% %%%%%%% configuration of schedule
% \CellHeight{14mm}
% \CellWidth{35mm}
% \TimeRange{8:30-18:30}
% \SubUnits{30}
% \BeginOn{Monday}
% \TextSize{\scriptsize}
% \FiveDay
%
% \NewAppointment{opening}{green}{black}
% \NewAppointment{courseOne}{courseOne}{white}
% \NewAppointment{courseTwo}{courseTwo}{black}
% \NewAppointment{courseThree}{courseThree}{black}
% \NewAppointment{coffee}{gray}{white}
% \NewAppointment{lunch}{maingray}{black}
% \NewAppointment{talk}{conferences}{black}
% \NewAppointment{exercises}{exercises}{black}
% \NewAppointment{excursion}{excursion}{white}
%
% %%%%% schedule
%
%
% \begin{schedule}[Split, September 9-13, 2024]
%   \opening{Opening}{}{M}{8:30-9:00}
%   \courseOne{Course 1}{Francisco Ortegón Gallego}{M,T,W}{9:00-10:30}
%   \courseTwo{Course 2}{J. Rafael Rodríguez Galván}{M,T}{11:00-12:30}
%   \courseThree{Course 3}{Malte Braack}{W,Th,F}{11:00-12:30}
%   \courseTwo{Course 2}{M. Victoria Redondo Neble}{F}{9:00-10:30}
%   \courseOne{Course 1}{Francisco Ortegón Gallego}{Th}{9:00-9:45}
%   \courseTwo{Course 2}{J. Rafael Rodríguez Galván}{Th}{9:45-10:30}
%   \courseThree{Course 3}{Malte Braack}{F}{11:00-12:30}
%   
%   \coffee{Cofee break}{}{M,T,W,Th,F}{10:30-11:00}
%   \lunch{Lunch}{}{M,T,W,Th,F}{12:30-14:00}
%   \coffee{Cofee break}{}{M,T,Th}{15:00-15:30}
%
%   \talk{Lecture 1}{Hermenegildo Borges de Oliveira}{M}{14:00-15:00}
%   \talk{Lecture 2}{Pending name}{M}{15:30-16:30}
%   \talk{Lecture 3}{Pending name}{T}{14:00-15:00}
%   \talk{Lecture 4}{Pending name}{Th}{14:00-15:00}
%
%   \exercises{Exercises I}{}{M}{16:30-18:30}
%   \exercises{Exercises II}{}{T}{15:30-18:30}
%   \exercises{Exercises III}{}{Th}{15:30-18:30}
%   \excursion{Excursion}{}{W}{14:00-18:30}
%   
% \end{schedule}

%%%%%%% configuration of schedule
\CellHeight{14mm}
\CellWidth{35mm}
\TimeRange{8:30-19:00}
\SubUnits{30}
\BeginOn{Monday}
\TextSize{\scriptsize}
\FiveDay

\NewAppointment{opening}{green}{black}
\NewAppointment{courseOne}{courseOne}{white}
\NewAppointment{courseTwo}{courseTwo}{black}
\NewAppointment{courseThree}{courseThree}{black}
\NewAppointment{coffee}{gray}{white}
\NewAppointment{lunch}{maingray}{black}
\NewAppointment{talk}{conferences}{black}
\NewAppointment{exercises}{exercises}{black}
\NewAppointment{excursion}{excursion}{white}

%%%%% schedule

% \newcommand{\tOpening}{8:30}
% \newcommand{\tZero}{9:00}


\newcommand{\tOpening}{8:30}
\newcommand{\tOne}{9:00}
\newcommand{\tTwo}{9:45}
\newcommand{\tCoffeeOne}{10:30}
\newcommand{\tThree}{11:00}
\newcommand{\tLunch}{12:30}
\newcommand{\tLecturesOne}{14:00}
\newcommand{\tCoffeeTwo}{15:00}
\newcommand{\tLecturesTwo}{15:30}
\newcommand{\tExercises}{16:30}
\newcommand{\tEnd}{18:30}

\begin{schedule}[Split, September 9-13, 2024]
  \opening{Opening}{}{M}{\tOpening-\tOne}

  % Courses
  \courseOne{Course 1}{Francisco Ortegón Gallego}{M,T,W}{\tOne-\tCoffeeOne}
  \courseOne{Course 1}{Francisco Ortegón Gallego}{Th}{\tOne-\tTwo}

  \courseTwo{Course 2}{J. Rafael Rodríguez Galván}{M,T}{\tThree-\tLunch}
  \courseTwo{Course 2}{J. Rafael Rodríguez Galván}{Th}{\tTwo-\tCoffeeOne}
  \courseTwo{Course 2}{M. Victoria Redondo Neble}{F}{\tOne-\tCoffeeOne}

  \courseThree{Course 3}{Malte Braack}{W,Th,F}{\tThree-\tLunch}
  
  % Lectures
  \talk{Lecture 1}{Pending name}{M}{\tLecturesOne-\tCoffeeTwo}
  \talk{Lecture 2}{Pending name}{M}{\tLecturesTwo-\tExercises}
  \talk{Lecture 3}{Pending name}{T}{\tLecturesOne-\tCoffeeTwo}
  \talk{Lecture 4}{Pending name}{T}{\tLecturesTwo-\tExercises}
  \talk{Lecture 5}{Pending name}{Th}{\tLecturesOne-\tCoffeeTwo}

  % Breaks and excursion
  \coffee{Cofee break}{}{M,T,W,Th,F}{\tCoffeeOne-\tThree}
  \coffee{Cofee break}{}{M,T,Th}{\tCoffeeTwo-\tLecturesTwo}
  \lunch{Lunch}{}{M,T,W,Th,F}{\tLunch-\tLecturesOne}
  \excursion{Excursion}{}{W}{\tLecturesOne-\tEnd}

  % % Exercises
  \exercises{Exercises I}{}{M}{\tExercises-\tEnd}
  \exercises{Exercises II}{}{T}{\tExercises-\tEnd}
  \exercises{Exercises III}{}{Th}{\tLecturesTwo-\tEnd}
  
\end{schedule}

%
\vspace{0.5cm}
\section{Schedule for Evaluation}

Those students interested in evaluation, for credits recognition, must submit a final group before Friday, \textbf{20th September}, 2024.

\end{document}



